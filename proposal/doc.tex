\documentclass{article}

\usepackage{amsthm}
\usepackage{amsmath}
\usepackage{cite}
\usepackage{listings}
\usepackage{multicol}
\usepackage{url}
\usepackage{xeCJK}
\setCJKmainfont{IPAMincho}

\setlength{\parindent}{4em}
\setlength{\parskip}{1em}

\title{Project Title Here}
\date{2019-03-01}
\author{Derick Anderson \\ anderson.de@husky.neu.edu
  \and Timothy Gillis \\ gillis.ti@husky.neu.edu }

\begin{document}

\pagenumbering{gobble}
\maketitle

\section*{Introduction}

We aim to learn to transliterate English to Japanese
without any parallel data.
That is to say,
given only monolingual corpora
and the tiniest insight about Japanese orthography
to learn how to represent English words in Japanese script.
Learning to transliterate with limited supervision has been well studied,
but so far as we can tell
there have been no attempts to learn without at least parallel data.
The key inspiration was the recent paper by
Conneau et al. \cite{Conneau2018WordTW}
in which word translations were learned with no parallel data
(or any other supervision).
Conneau et al. did not touch Japanese,
so we hope to first replicate their work on the Japanese-English language pair.
We then hope to discover an algorithm to learn to transliterate
using those (probably noisy) results as training data.

Derick has a personal interest in Japanese, transliteration, and both together.
% Tim - any note about why you're interested?
It will be an interesting,
even if a little incremental,
step forward if we can manage to transliterate
without parallel data.

\section*{Background}

\subsection*{Transliteration}

Transliteration is representing words
in a script or orthographic style
other than that with which they were originally represented.
Between close scripts like the Old English alphabet and the modern English alphabet
this can be a mechanical character to character mapping:
e.g. ``þe olde'' to ``ye old''.
Between distant scripts
like the English alphabet and Japanese katakana
the task becomes more difficult.
Translation includes transliteration as a subtask,
although the relationship between the two
depends on the language pair, intended audience, and other factors.

In the literature on machine transliteration there is a distinction
between generative transliteration and transliteration extraction.
Generative transliteration learns a function for transliterating;
transliteration extraction just identifies pairs of strings
to be added to a transliteration dictionary
\cite{Karimi:2011:MTS:1922649.1922654}.
In this paper we will be doing generative transliteration.

In translating into both English and Japanese
probably the most common use of transliteration
is representing foreign proper nouns.
Because proper nouns are an open and varied class
there has been a lot of work on unsupervised learning to transliterate them
(e.g. \cite{Tao2006UnsupervisedNE}),
maybe for use in machine translation (e.g. \cite{Durrani2014IntegratingAU})
or cross-language information retrieval (e.g. \cite{10.1007/978-3-642-40087-2_29}).
A limitation of our proposed approach
is that we will not pay any special attention to proper nouns,
which often have their own conventions for transliteration.
% maybe we will want to?

The use of transliteration most relevant to this project
is the borrowing of loan words from foreign languages.
These can be common nouns (bed, ベッド),
verbs (join, ジョイン),
adjectives (sexy, セクシー),
or even sentence pieces (let's!, レッツ!).

\subsection*{The FB Research Paper} % not actual section title

\section*{Proposed Approach}

\section*{Individual Tasks}

\bibliography{doc}{}
\bibliographystyle{plain}
\end{document}
%%% Local Variables:
%%% mode: latex
%%% TeX-engine: xetex
%%% TeX-master: t
%%% End:
