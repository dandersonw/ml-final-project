\documentclass{article}

\usepackage{amsthm}
\usepackage{amsmath}
\usepackage{cite}
\usepackage{listings}
\usepackage{multicol}
\usepackage{url}

\setlength{\parindent}{4em}
\setlength{\parskip}{1em}

\title{Project Title Here}
\date{2019-03-01}
\author{Derick Anderson \\ anderson.de@husky.neu.edu
  \and Timothy Gillis \\ gillis.ti@husky.neu.edu }

\begin{document}

\pagenumbering{gobble}
\maketitle

\section*{Introduction}

We aim to learn to transliterate English to Japanese
without any parallel data.
That is to say,
given only monolingual corpora
and the tiniest insight about Japanese orthography
to learn how to represent English words in Japanese script.
Learning to transliterate with limited supervision has been well studied,
but so far as we can tell
there have been no attempts to learn without at least parallel data.
The key inspiration was the recent paper by
Conneau et al. \cite{Conneau2018WordTW}
in which word translations were learned with no parallel data
(or any other supervision).
Conneau et al. did not touch Japanese,
so we hope to first replicate their work on the Japanese-English language pair.
We then hope to discover an algorithm to learn to transliterate
using those (probably noisy) results as training data.

Derick has a personal interest in Japanese, transliteration, and both together.
% Tim - any note about why you're interested?
It will be an interesting,
even if a little incremental,
step forward if we can manage to transliterate
without parallel data.

\section*{Background}

\section*{Proposed Approach}

\section*{Individual Tasks}

\bibliography{doc}{}
\bibliographystyle{plain}
\end{document}
%%% Local Variables:
%%% mode: latex
%%% TeX-master: t
%%% End:
