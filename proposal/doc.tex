\documentclass{article}

\usepackage{amsthm}
\usepackage{amsmath}
\usepackage{cite}
\usepackage{listings}
\usepackage{multicol}
\usepackage{url}
\usepackage{xeCJK}
\setCJKmainfont{IPAMincho}

\setlength{\parindent}{4em}
\setlength{\parskip}{1em}

\title{Project Title Here}
\date{2019-03-01}
\author{Derick Anderson \\ anderson.de@husky.neu.edu
  \and Timothy Gillis \\ gillis.ti@husky.neu.edu }

\begin{document}

\pagenumbering{gobble}
\maketitle

\section*{Introduction}

We aim to learn to transliterate English to Japanese
without any parallel data.
That is to say,
given only monolingual corpora
and the tiniest insight about Japanese orthography
to learn how to represent English words in Japanese script.
Learning to transliterate with limited supervision has been well studied,
but so far as we can tell
there have been no attempts to learn without at least parallel data.
The key inspiration was the recent paper by
Conneau et al. \cite{Conneau2018WordTW}
in which word translations were learned with no parallel data
(or any other supervision).
Conneau et al. did not touch Japanese,
so we hope to first replicate their work on the Japanese-English language pair.
We then hope to discover an algorithm to learn to transliterate
using those (probably noisy) results as training data.

Derick has a personal interest in Japanese, transliteration, and both together.
% Tim - any note about why you're interested?
It will be an interesting,
even if a little incremental,
step forward if we can manage to transliterate
without parallel data.

\section*{Background}

\subsection*{Transliteration}

Transliteration is representing words
in a script or orthographic style
other than that with which they were originally represented.
Between close scripts like the Old English alphabet and the modern English alphabet
this can be a mechanical character to character mapping:
e.g. ``þe olde'' to ``ye olde''.
Between distant scripts
like the English alphabet and Japanese katakana
the task becomes more difficult.
Translation includes transliteration as a subtask,
although the relationship between the two
depends on the language pair, intended audience, and other factors.

In the literature on machine transliteration there is a distinction
between generative transliteration and transliteration extraction.
Generative transliteration learns a function for transliterating;
transliteration extraction just identifies pairs of strings
to be added to a transliteration dictionary
\cite{Karimi:2011:MTS:1922649.1922654}.
In this paper we will be doing generative transliteration.

In translating into both English and Japanese
probably the most common use of transliteration
is representing foreign proper nouns.
Because proper nouns are an open and varied class
there has been a lot of work on unsupervised learning to transliterate them
(e.g. \cite{Tao2006UnsupervisedNE}),
maybe for use in machine translation (e.g. \cite{Durrani2014IntegratingAU})
or cross-language information retrieval (e.g. \cite{10.1007/978-3-642-40087-2_29}).
A limitation of our proposed approach
is that we will not pay any special attention to proper nouns,
which often have their own conventions for transliteration.
% maybe we will want to?

The use of transliteration most relevant to this project
is the borrowing of words from foreign languages.
Like proper nouns,
some words may not have equivalents in every language,
and rather than create a neologism from native material (e.g. as done in Icelandic)
a foreign word might be adopted and used.

\subsection*{The Japanese-English Language Pair}

In modern orthography
Japanese represents foreign words
by approximating the pronunciation of the foreign word
in a syllabary \footnote{In a syllabary each character represents a syllable,
although katakana has some exceptions by that definition.} called katakana.
Native Japanese words and loanwords from Chinese
\footnote{Words of Sinitic origin are so common in Japanese that they not considered
  foreign in the same sense as words from English.}
are usually represented in scripts besides katakana:
the hiragana syllabary and kanji
\footnote{``Kanji'' is the Japanese name for Chinese characters.}.
The approximation of pronunciation is really quite approximate;
as Japanese has a relatively limited phonetic inventory (range of sounds used)
not all English sounds can be represented.
In the face of the complicated correspondence
between English spelling and English pronunciation
(to say nothing of international English variants)
Japanese people sometimes transliterate based on the spelling of a word
rather than the pronunciation.
An ideal system,
therefore,
will probably need to consider both phonetics and spelling.

Japanese has taken many loan words from English.
These can be common nouns (bed, ベッド),
verbs (join, ジョイン),
adjectives (sexy, セクシー),
or even sentence pieces (let's!, レッツ!).
Many (most?) loan words
are used in the same ways as their English equivalents,
although the meaning of some has diverged.
Japanese has also taken loanwords from other languages
that are written in katakana,
some of which may conflict with English words.
An example is ナトリウム,
translated from the Latin natrium,
meaning sodium (consider the Na elemental abbreviation).
The hope is that clever matching of English words
will allow useful transcription pairs to be found.

\subsection*{The FB Research Paper} % not actual section title

\section*{Proposed Approach}

\subsection*{Transliteration}

In order to not be blocked on
the completion of the word translation portion of the project,
we will substitute in some ground-truth data
while developing the transliteration half of the project.
Conveniently,
Facebook Research provides a ground truth English-Japanese dictionary
of the same sort as is used for their evaluation of word translation,
available on their GitHub
\footnote{https://github.com/facebookresearch/MUSE}.
It contains about 14,000 translations of an English word to a katakana string.
There is a slightly larger dataset available here
\footnote{https://github.com/eob/english-japanese-transliteration}
that we might choose to use.
CMU makes available a large dataset for English pronunciation
\footnote{http://www.speech.cs.cmu.edu/cgi-bin/cmudict}.

Considering just the bird's eye view of the task
as transforming a sequence of characters to another sequence of characters,
our first thought was to use a sequence-to-sequence model
like the well known Google neural machine translation model
\cite{Wu2016GooglesNM}.
Not surprisingly,
some people at Google did try that \cite{Rosca2016SequencetosequenceNN}
and found that the results were highly competitive.
Since attentional sequence-to-sequence models \cite{Bahdanau2015NeuralMT}
are so well known,
we won't describe the details here.

Treating words as just sequences of characters
means that we diverge from a lot of transliteration methods
in ignoring pronunciation.
The main motivation for that
is that we won't necessarily have pronuncation information for translation-pairs
that we get out of the word translation half of the project.
Something of a stretch goal is to validate the intuition
that considering both pronunciation and spelling will be helpful to transliteration.
Coincidentally,
the authors of the neural transliteration paper \cite{Rosca2016SequencetosequenceNN}
note that that seems like a good idea.
Considering pronunciation would be achieved by
training a sequence-to-sequence model to predict
the phoneme strings in the CMU pronunciation dataset.
The input to the transliteration model could then be
(most simply) appended with the predicted phoneme string.

For implementation,
Derick will use TensorFlow as that is the framework he is comfortable with.

Our personal opinion is that evaluating the correctness
of a transliterator is not that straightforward.
Nevertheless,
we will use a few straightforward measures
that appear in the transliteration literature.
The first is simply accuracy:
for what proportion of the input words
does the transliterator return the correct transliteration.
The second is Levenstein distance:
the number of simple edits necessary to change the output to
the correct transliteration
\footnote{https://en.wikipedia.org/wiki/Levenshtein_distance}.
The third is mean reciprocal rank (MRR):
a measure of how far down the list of possible results
(ordered in direction of decreasing probability)
the correct answer is
\footnote{https://en.wikipedia.org/wiki/Mean_reciprocal_rank}.
If we incorporate pronunciation or other features
we will perform a simple ablation study
to determine which features were useful.

\section*{Individual Tasks}

\bibliography{doc}{}
\bibliographystyle{plain}
\end{document}
%%% Local Variables:
%%% mode: latex
%%% TeX-engine: xetex
%%% TeX-master: t
%%% End:
